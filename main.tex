\section{Introduction}
This summary consolidates U.S. spent nuclear fuel (SNF) and high-level waste inventory information from federal databases 
and national reports to support isotopic composition analysis and fuel-cycle modeling. It provides 
(1) current commercial SNF inventory, 
(2) DOE-managed SNF from research and defense programs, 
(3) research reactor SNF, and (4) reprocessing waste inventories.

\begin{figure}[H]
    \begin{center}
        \includegraphics[width=1.0\textwidth]{figures/US_map.png}
    \end{center}
    \caption{Sites Storing Spent Nuclear Fuel and Reprocessing Waste at the End of 2022}
    \label{fig1}
\end{figure}

At the end of 2022, the U.S. inventory of spent nuclear fuel (SNF) and primary reprocessing waste was distributed across more than 100 sites in 39 states. 
These include 73 commercial nuclear power reactor and ISFSI sites, 6 DOE facilities with SNF or research reactors, 
and 28 research and development (R\&D) sites encompassing universities, government laboratories, and commercial research centers. 
Four major locations manage high-level waste (HLW) and vitrified reprocessing waste, including 3 DOE facilities and one commercial site at West Valley, NY
\begin{table}[H]
    \begin{center}
        \includegraphics[width=1.0\textwidth]{figures/intro_table.png}
    \end{center}
    \caption{U.S. SNF and Reprocessing Waste Inventory Summary as of December 31, 2022}
    \label{tab1}
\end{table}

Commercial SNF inventories include permanently discharged PWR and BWR assemblies reported through 2022.
Quantities exclude reprocessed fuel from the West Valley and Fort St. Vrain reactors. 
DOE totals include SNF from research, naval, and production programs. 
Other non-commercial sources include university and government research reactors. 
Reprocessing waste totals reflect vitrified HLW canisters produced through 2022.
%%%%%%%%%%%%%%%%%%%%%%%%%%%%%%%%%%%%%%%%%%%%%%%%%%%%%%%%%%%%%%%%%%%%%%%%%%%%%%%%%%%%%%%%%%%%%%%%%%%%%%%%%%%%%%%%%%%%%%%%%%%%%%%%%%%%%%%%%%%

%%%%%%%%%%%%%%%%%%%%%%%%%%%%%%%%%%%%%%%%%%%%%%%%%%%%%%%%%%%%%%%%%%%%%%%%%%%%%%%%%%%%%%%%%%%%%%%%%%%%%%%%%%%%%%%%%%%%%%%%%%%%%%%%%%%%%%%%%%%
\section{U.S. Commercial Spent Nuclear Fuel Inventory}
Since 1960, the U.S. has built 131 nuclear power reactors (NPRs) for civilian electricity generation, 
excluding experimental and non-power units. Nine early demonstration reactors (e.g., Peach Bottom 1, Shippingport, Fermi 1) 
have been decommissioned, and their SNF is managed by DOE. The Fort St. Vrain gas-cooled reactor’s fuel is also under DOE custody 
at Idaho National Laboratory (INL).\\

Of the remaining reactors, 121 are light-water reactors (LWRs). One (Shoreham) never reached full operation, 
and Three Mile Island Unit 2 was disabled in 1979—both managed by DOE. As of 2022, 92 reactors operate and 27 are permanently shut down. 
Each plant includes one or more reactors and typically a co-located Independent Spent Fuel Storage Installation (ISFSI), 
which becomes the sole facility after decommissioning.\\

To organize analysis, sites are grouped by reactor status and storage mode:

\begin{itemize}
    \item Group A: All units permanently shut down.
    \item Group B: Mixed—operating and shutdown units.
    \item Group C: All operating units.
    \item Group F: Away-from-reactor ISFSI (Morris, IL)
\end{itemize}

A suffix denotes storage configuration (1 = dry only, 2 = mixed, 3 = wet only). For example, Yankee Rowe is Group A1, 
while Surry is Group C2. Two Diablo Canyon reactors plan early shutdowns before 2030.\\

\begin{table}[H]
    \begin{center}
        \includegraphics[width=1.0\textwidth]{figures/sec2.1_table.png}
    \end{center}
    \caption{Spent Nuclear Fuel Inventory by Reactor Group/Subgroup (As of 12/31/2022)}
\end{table}
%%%%%%%%%%%%%%%%%%%%%%%%%%%%%%%%%%%%%%%%%%%%%%%%%%%%%%%%%%%%%%%%%%%%%%%%%%%%%%
%\subsection{Reactor Grouping and Site Classification}
Inventory and fuel design data originate from DOE’s GC-859 Nuclear Fuel Data Survey, which catalogs assembly dimensions, 
enrichment, and burnup characteristics used throughout the report.\\
At the end of 2022, approximately 91 000 MTU of commercial SNF had been discharged. 
The discharge totals reported via GC-859 represent only the initial discharge locations and do not account for subsequent transfers.\\

\begin{itemize}
    \item Before 2000, limited quantities were reprocessed at West Valley, NY, and five reactors shipped fuel to the Morris, IL
     away-from-reactor pool storage site.
    \item bout 73 MTU of SNF was transferred to DOE for research and demonstration programs 
    (e.g., rod consolidation, dry storage testing, vitrification). Roughly 68 MTU remain in DOE storage today.
    \item SNF from Fort St. Vrain and Three Mile Island Unit 2 is also managed by DOE at Idaho National Laboratory.
    \item Since 2000, nearly all discharged SNF has remained at the originating reactor sites, stored in pools or dry casks.
    \item As of 2022, 73 commercial sites, including the Morris facility, hold the national inventory. More SNF is now stored in dry systems than in pools
\end{itemize}

\begin{table}[H]
    \begin{center}
        \includegraphics[width=1.0\textwidth]{figures/sec2.2_tab.png}
    \end{center}
    \caption{Current Inventory at NPR sites by Storage Method as of December 2022}
\end{table}


\begin{figure}[H]
    \begin{center}
        \includegraphics[width=1.0\textwidth]{figures/fig2.1.png}
    \end{center}
    \caption{Nuclear Power Reactor and ISFSI Sites (non-DOE) Currently Storing SNF}
\end{figure}

\begin{figure}[H]
    \begin{center}
        \includegraphics[width=1.0\textwidth]{figures/fig2.11.png}
    \end{center}
    \caption{Dry SNF Storage Casks Loaded at Nuclear Power Reactor Sites}
\end{figure}

Fuel characteristics derived from GC-859 show:
\begin{itemize}
    \item verage burnup increasing steadily since the 1970s to about 45–50 GWd/MT.
    \item Corresponding enrichment rising from 2\% to 4.5\% U-235.
\end{itemize}

\begin{figure}[H]
    \begin{center}
        \includegraphics[width=1.0\textwidth]{figures/assemblies.png}
    \end{center}
    \caption{Burn-up (GWd/MTHM) \& Initial Enrichment (\% U-235) by Number of Assemblies of SNF Through December 2022}
\end{figure}

\begin{figure}[H]
    \begin{center}
        \includegraphics[width=1.0\textwidth]{figures/burnup_enrichment.png}
    \end{center}
    \caption{Average Annual Burn-up (GWd/MT) and Enrichment (U-235\%)}
\end{figure}

The figures below show the decay heat behavior of representative spent fuel assemblies at typical U.S. discharge burnups (PWR: 40 and 60\~GWd/MTU; BWR: 30 and 50\~GWd/MTU).

\begin{table}[H]
    \begin{center}
        \includegraphics[width=1.0\textwidth]{figures/pwr_table.png}
    \end{center}
    \caption{PWR 60 GWd/MT Spent Nuclear Fuel Decay Heat}
\end{table}

\begin{figure}[H]
    \begin{center}
        \includegraphics[width=1.0\textwidth]{figures/pwr_plot.png}
    \end{center}
    \caption{PWR 60 GWd/MT Spent Nuclear Fuel Decay Heat}
\end{figure}

\begin{table}[H]
    \begin{center}
        \includegraphics[width=1.0\textwidth]{figures/bwr_table.png}
    \end{center}
    \caption{BWR 50 GWd/MT Spent Nuclear Fuel Decay Heat}
\end{table}

\begin{figure}[H]
    \begin{center}
        \includegraphics[width=1.0\textwidth]{figures/bwr_plot.png}
    \end{center}
    \caption{BWR 50 GWd/MT Spent Nuclear Fuel Decay Heat}
\end{figure}


While commercial power reactors account for nearly all spent fuel generated in the United States by mass, 
the Department of Energy manages a separate inventory originating from research, defense, and demonstration reactors.
%%%%%%%%%%%%%%%%%%%%%%%%%%%%%%%%%%%%%%%%%%%%%%%%%%%%%%%%%%%%%%%%%%%%%%%%%%%%%%%%%%%%%%%%%%%%%%%%%%%%%%%%%%%%%%%%%%
\section{SNF at DOE locations}

The U.S. Department of Energy (DOE) manages spent nuclear fuel (SNF) that is not part of the commercial nuclear power reactor fleet.
This includes fuel from defense production reactors, DOE-sponsored research reactors, early demonstration power reactors, 
fuel used in research and development programs, and naval reactors.\\

DOE SNF comes from a diverse set of reactors, including production and isotope-generation reactors, 
national laboratory research reactors, university and government research reactors, and early power demonstration reactors
(e.g., Shippingport, Peach Bottom 1). SNF is highly heterogeneous, ranging from depleted uranium fuel to highly enriched 
(>93\% U-235) test reactor fuel. DOE holds fuel in hundreds of distinct fuel types—over 229,000 individual pieces.\\

As of the end of 2022, DOE managed approximately 2,446 metric tons of heavy metal (MTHM) of SNF (excluding naval fuel), 
tracked in the Spent Fuel Database (SFD). For analysis purposes, this inventory has been grouped into 34 standardized 
categories based on fuel matrix, cladding type, cladding condition, and enrichment—parameters that affect storage, degradation, 
and repository performance.\\

DOE estimates radionuclide inventories using template models based on fuel type and decay time. 
The estimated total activity for DOE SNF in 2030 is:
\begin{itemize}
    \item 96 million Ci (nominal)
    \item 195 million Ci (bounding/highest case)
\end{itemize}

\begin{table}[H]
    \begin{center}
        \includegraphics[width=1.0\textwidth]{figures/doe_projection.png}
    \end{center}
    \caption{Spent Nuclear Fuel Canister Decay Heat in 2030 [NSNFP, 2023]}
\end{table}

In 1995, DOE consolidated SNF storage at three sites, where it remains today:
\begin{itemize}
    \item Hanford (WA): 2,128 MTHM (largest share, stored in Multi-Canister Overpacks)
    \item Idaho National Laboratory (ID): \~270 MTHM
    \item Savannah River Site (SC): \~27 MTHM
\end{itemize}

Most SNF is stored in dry canisters; DOE has standard disposal canister designs but has not yet packaged fuel into those canisters.\\

DOE currently manages 173.6 MTHM of commercial-origin SNF. DOE also manages some fuel originally discharged from commercial nuclear power reactors for research, 
testing, and post-irradiation examination. This includes:

\begin{itemize}
    \item Three Mile Island Unit 2 core debris.
    \item Fort St. Vrain High-Temperature Gas-Cooled Reactor fuel.
    \item Fuel transferred for research/vitrification testing.
\end{itemize}

\begin{table}[H]
    \begin{center}
        \includegraphics[width=1.0\textwidth]{figures/doe.png}
    \end{center}
    \caption{Canister Decay Heat Characteristics of NPR Origin SNF in DOE Possession}
\end{table}

The Naval Nuclear Propulsion Program generates SNF from nuclear-powered submarines, 
aircraft carriers, and land-based training reactors.

\begin{itemize}
    \item Current naval SNF inventory: \~40 MTHM, projected to be <65 MTHM by 2035.
    \item Uses highly enriched uranium, resulting in very low transuranic buildup compared to commercial SNF.
    \item Stored in engineered naval canisters designed to meet repository acceptance limits.
    \item About 201 naval fuel canisters are currently loaded and stored at INL.
\end{itemize}

\begin{table}[H]
    \begin{center}
        \includegraphics[width=1.0\textwidth]{figures/naval.png}
    \end{center}
    \caption{Naval SNF Canister Decay Heat}
\end{table}


\section{SNF at other sites}
In addition to commercial power reactors and DOE facilities, a small amount of spent nuclear fuel (SNF) exists at other U.S. sites,
including university research reactors, government agency reactors, and commercial research and development centers. 
These inventories are tracked through the DOE Spent Fuel Database and total only \~1.35 metric tons of heavy metal (MTHM),
a very small quantity compared to commercial and the DOE managed SNF.\\

Approximately 20 university research reactors are actively operating across the United States. 
They operate at very low power levels (from <1 watt to 10 MW) and use fuel at such low burnup that refueling is infrequent or 
unnecessary during their operational lifetime. When these reactors eventually discharge spent fuel, 
it is transferred to DOE storage facilities, primarily at Idaho National Laboratory (INL) or the Savannah River Site (SRS), 
where it becomes part of the DOE-managed SNF inventory described. 
A small number of very low-power critical facilities (e.g., AGN-201 reactors and the Rensselaer Critical Facility) are not expected to ever discharge fuel.

\begin{table}[H]
    \begin{center}
        \includegraphics[width=1.0\textwidth]{figures/universities.png}
    \end{center}
    \caption{University Research Reactors}
\end{table}

Table 4.2 lists research reactors operated by other government organizations. Permanently
discharged SNF from these reactors is generally sent to either SRS or INL, and the SNF is managed
by DOE and included in the inventory discussed earlier.
\begin{table}[H]
    \begin{center}
        \includegraphics[width=1.0\textwidth]{figures/government.png}
    \end{center}
    \caption{Other Government Agency Research Reactors SNF}
\end{table}

Table 4.3 lists commercial research and development centers. Three sites have reactors while the
BWX Technologies site in Virginia is a fuel cycle research center conducting SNF destructive
examinations among other activities.
\begin{table}[H]
    \begin{center}
        \includegraphics[width=1.0\textwidth]{figures/commercial_research.png}
    \end{center}
    \caption{Commercial Research and Development Centers SNF}
\end{table}


\section{Reprocessing Waste}
In addition to spent nuclear fuel (SNF), the United States also manages high-level radioactive waste (HLW) 
resulting from reprocessing—the chemical separation of plutonium or uranium from irradiated fuel. 
Reprocessing historically took place at three DOE sites (Hanford, Idaho National Laboratory, and Savannah River Site) 
and at one former commercial reprocessing plant (West Valley, NY).\\

Reprocessing waste exists in different physical forms depending on the treatment method:
\begin{itemize}
    \item Liquid HLW in underground tanks (historical storage)
    \item Vitrified waste (liquid waste converted into stable glass inside stainless-steel canisters)
    \item Calcined waste (granular solid waste produced by high-temperature drying)
    \item Metal/ceramic waste from electrochemical treatment of sodium-bonded fuel at INL
\end{itemize}

At DOE sites, vitrification is the primary stabilization strategy. Savannah River Site has been vitrifying waste since 1996 and, 
as of Dec. 31, 2022, has produced 4,346 glass waste canisters. INL holds \~4,400 m³ of calcined waste from past reprocessing 
campaigns. Hanford still stores \~220,000 m³ of liquid HLW and is constructing a vitrification plant.\\

Hanford also stores cesium and strontium capsules (1,936 total), originally removed from liquid waste for isotope uses; 
their total radioactivity has declined from \~109 million curies to \~42 million curies through decay.\\

Electrochemical processing at INL (for sodium-bonded SNF) is treating up to 60 MTHM, producing both ceramic and metallic waste forms.\\

The combined radionuclide inventory of DOE reprocessing waste corresponds to approximately 1.3 million watts of decay heat, 
with more than 99\% of vitrified canisters generating <1 kW.\\

At West Valley (NY), the only U.S. commercial reprocessing plant (operated 1966–1972), 
640 MTHM of SNF was processed, generating \~2,500 m³ of liquid HLW. This waste was vitrified into 278 canisters (stored on site).
%%%%%%%%%%%%%%%%%%%%%%%%%%%%%%%%%%%%%%%%%%%%%%%%%%%%%%%%%%%%%%%%%%%%%%%%%%%%%%%%%%%%%%%%%%%%%%%%%%%%%%%%%%%%%%%%%%%%%%%%%%%%%%%%%%%%%%%%%%%%%%%%%%%
\section{Conclusion}
The U.S. spent nuclear fuel inventory is dominated by commercial LWR SNF stored at reactor sites, 
with a smaller but more diverse DOE inventory and isolated reprocessing waste. 
Increasing burnup and enrichment trends influence isotopic composition at discharge, 
which is critical for fuel cycle modeling, decay heat analysis, and storage/disposal system design.\\

Key Findings:

\begin{itemize}
    \item Total commercial SNF discharged through 2022: $\approx$ 91,000 MTU
    \item DOE-managed SNF (non-commercial): $\approx$ 2,446 MTU
    \item SNF at universities / R\&D sites: $\approx$ 1.35 MTU
    \item High-level reprocessing waste: SRS vitrified canisters: 4,346 + West Valley vitrified canisters: 278

\end{itemize}

trying bib..~\cite{rykhlevskii_online_2017}