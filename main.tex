\section{Introduction}
    This summary consolidates U.S. spent nuclear fuel (SNF) and high-level waste inventory information from federal databases 
    and national reports to support isotopic composition analysis and fuel-cycle modeling. It provides 
    (1) current commercial SNF inventory, 
    (2) DOE-managed SNF from research and defense programs, 
    (3) research reactor SNF, and (4) reprocessing waste inventories.

    \begin{figure}[H]
        \begin{center}
            \includegraphics[width=1.0\textwidth]{figures/US_map.png}
        \end{center}
        \caption{Sites Storing Spent Nuclear Fuel and Reprocessing Waste at the End of 2022}
        \label{fig1}
    \end{figure}

    At the end of 2022, the U.S. inventory of spent nuclear fuel (SNF) and primary reprocessing waste was distributed across more than 100 sites in 39 states. 
    These include 73 commercial nuclear power reactor and ISFSI sites, 6 DOE facilities with SNF or research reactors, 
    and 28 research and development (R\&D) sites encompassing universities, government laboratories, and commercial research centers. 
    Four major locations manage high-level waste (HLW) and vitrified reprocessing waste, including 3 DOE facilities and one commercial site at West Valley, NY
    \begin{table}[H]
        \begin{center}
            \includegraphics[width=1.0\textwidth]{figures/intro_table.png}
        \end{center}
        \caption{U.S. SNF and Reprocessing Waste Inventory Summary as of December 31, 2022}
        \label{tab1}
    \end{table}

    Commercial SNF inventories include permanently discharged PWR and BWR assemblies reported through 2022.
    Quantities exclude reprocessed fuel from the West Valley and Fort St. Vrain reactors. 
    DOE totals include SNF from research, naval, and production programs. 
    Other non-commercial sources include university and government research reactors. 
    Reprocessing waste totals reflect vitrified HLW canisters produced through 2022.

\section{U.S. Commercial Spent Nuclear Fuel Inventory} 


 Commercial nuclear power reactors have operated in the United States since 1960, during which time a total of
 131 civilian nuclear power reactors (NPRs) were constructed,excluding several experimental or non-power demonstration units. 
 Nine of the early demonstration reactors, including Peach Bottom Unit 1, Shippingport, and Fermi 1, have been fully decommissioned, 
 and the remaining spent nuclear fuel (SNF) from these reactors is managed by the Department of Energy (DOE). The Fort St. 
 Vrain high-temperature gas-cooled reactor, although also decommissioned, has its SNF under DOE custody at 
 the Idaho National Laboratory (INL). Of the remaining fleet, 121 are light-water reactors (LWRs). 
 Shoreham never entered commercial service, and Three Mile Island Unit 2 was permanently disabled in 1979, 
 with both now under DOE management. As of the end of 2022, 92 reactors remain operational and 27 are permanently shut down. 
 Nearly every plant hosts a co-located Independent Spent Fuel Storage Installation (ISFSI), which becomes the primary 
 storage facility after plant decommissioning.\\

    To facilitate inventory analysis, reactor sites are assigned to four categories based on operational status and onsite 
    storage configuration. Group~A consists of sites where all reactors have been permanently shut down, while Group~B 
    contains sites with a mixture of operating and shutdown units. Group~C includes sites where all reactor units remain 
    operational, and Group~F consists of away-from-reactor ISFSI facilities such as the Morris, Illinois wet-storage pool. 
    A numeric suffix is used to indicate whether SNF storage at the site is entirely dry (1), a mixture of wet and dry storage (2), 
    or entirely wet (3). For example, Yankee Rowe is classified as Group~A1 because all SNF at the site is stored in dry casks, 
    while Surry is classified as Group~C2 because it contains both pool and dry storage systems. Diablo Canyon Units 1 and 2 are 
    currently classified under Group~C but are expected to transition to shutdown status before 2030.



    \begin{table}[H]
        \begin{center}
            \includegraphics[width=1.0\textwidth]{figures/sec2.1_table.png}
        \end{center}
        \caption{Spent Nuclear Fuel Inventory by Reactor Group/Subgroup (As of 12/31/2022)}
    \end{table}

    Table~1 summarizes the distribution of commercial SNF by reactor category, illustrating the growth of Group~A sites as more 
    reactors retire. This grouping framework underpins the spatial and temporal distribution of discharged fuel and the corresponding 
    storage infrastructure needed across the United States.\\

    Commercial SNF data originate primarily from the DOE’s GC-859 Nuclear Fuel Data Survey, which provides assembly-level 
    discharge information including assembly type, physical dimensions, enrichment, and burnup. As of December 2022, 
    approximately 91{,}000~MTU of commercial SNF had been discharged nationally. These discharge records reflect only the 
    initial removal from reactor cores and do not include subsequent transfers. Historically, small quantities of fuel were shipped 
    to other facilities prior to the year 2000, including limited reprocessing at West Valley, New York, and transfers to the Morris, 
    Illinois away-from-reactor wet storage pool. Approximately 73~MTU of fuel was transferred to DOE facilities for research programs 
    involving rod consolidation, dry storage demonstration, and vitrification studies; 68~MTU of this material remains in DOE custody 
    today. SNF from Fort St. Vrain and Three Mile Island Unit 2 is also stored at INL. Since 2000, however, virtually all discharged 
    commercial SNF has remained onsite at the generating reactor facilities. In total, 73 commercial locations, including Morris, 
    currently store SNF. Notably, as of 2021, the national inventory stored in dry casks surpassed the inventory remaining in spent 
    fuel pools, reflecting both aging reactor infrastructure and the expansion of dry storage systems across the fleet.

    \begin{table}[H]
        \begin{center}
            \includegraphics[width=1.0\textwidth]{figures/sec2.2_tab.png}
        \end{center}
        \caption{Current Inventory at NPR sites by Storage Method as of December 2022}
    \end{table}

    Table~2 presents the inventory distribution between wet and dry storage across all NPR sites. 
    This table highlights the ongoing shift from pool storage to dry cask systems, driven by pool capacity limits, 
    plant shutdowns, and long-term storage planning in the absence of a federal repository.

    \begin{figure}[H]
        \begin{center}
            \includegraphics[width=1.0\textwidth]{figures/fig2.1.png}
        \end{center}
        \caption{Nuclear Power Reactor and ISFSI Sites (non-DOE) Currently Storing SNF}
    \end{figure}

    Figure~1 depicts the geographic distribution of commercial reactor and ISFSI sites currently storing SNF. 
    The map highlights regions with particularly dense inventories, including the northeastern United States, the Midwest, 
    and the West Coast. This spatial distribution is an important consideration for future consolidation, transportation planning, 
    and repository siting, as it reflects the historical locations of the nation’s LWR fleet.

    \begin{figure}[H]
        \begin{center}
            \includegraphics[width=1.0\textwidth]{figures/fig2.11.png}
        \end{center}
        \caption{Dry SNF Storage Casks Loaded at Nuclear Power Reactor Sites}
    \end{figure}

    Figure~2 illustrates the cumulative number of dry storage casks loaded across the commercial reactor fleet. 
    The increasing trend reflects both ongoing reactor operations and the migration of older assemblies from pools to dry storage. 
    The widespread use of dry casks underscores the importance of long-term interim storage in national waste management planning.\\

    GC-859 data also provide insights into fuel characteristics at discharge. Over time, average burnup has increased from 
    early values below 20~GWd/MT in the 1970s to approximately 45--50~GWd/MT today. Correspondingly, initial enrichment has 
    risen from about 2\% U-235 in early fuel designs to nearly 4.5\% in modern PWR and BWR assemblies. These changes reflect 
    improvements in core design, regulatory allowances, and utility strategies aimed at extending fuel cycle lengths and 
    improving economic performance.

    \begin{figure}[H]
        \begin{center}
            \includegraphics[width=1.0\textwidth]{figures/assemblies.png}
        \end{center}
        \caption{Burn-up (GWd/MTHM) \& Initial Enrichment (\% U-235) by Number of Assemblies of SNF Through December 2022}
    \end{figure}

    Figure~3 presents the distribution of burnup and enrichment for discharged assemblies, demonstrating a clear shift toward 
    higher burnup fuel across the fleet. This trend influences isotopic composition at discharge, particularly increasing 
    plutonium and minor actinide production.

    \begin{figure}[H]
        \begin{center}
            \includegraphics[width=1.0\textwidth]{figures/burnup_enrichment.png}
        \end{center}
        \caption{Average Annual Burn-up (GWd/MT) and Enrichment (U-235\%)}
    \end{figure}

    Figure~4 shows the annual average enrichment and burnup of U.S. reactor discharges from 
    1968 to 2022. The steady upward trend in both parameters reflects technical and regulatory evolution in fuel management, 
    including the adoption of higher-enriched fuel and increased cycle lengths.\\

    Representative burnup cases, drawn from the UFDC Inventory Report, provide insight into decay heat behavior. 
    These cases include PWR fuel discharged at 40 and 60~GWd/MT and BWR fuel discharged at 30 and 50~GWd/MT—typical 
    values bounding the burnup range of modern LWR fuel. These representative isotopic compositions are commonly used in fuel 
    cycle modeling tools such as Cyclus, ORIGEN, and UNF-ST\&DARDS.

    \begin{table}[H]
        \begin{center}
            \includegraphics[width=1.0\textwidth]{figures/pwr_table.png}
        \end{center}
        \caption{PWR 60 GWd/MT Spent Nuclear Fuel Decay Heat}
    \end{table}

    \begin{figure}[H]
        \begin{center}
            \includegraphics[width=1.0\textwidth]{figures/pwr_plot.png}
        \end{center}
        \caption{PWR 60 GWd/MT Spent Nuclear Fuel Decay Heat}
    \end{figure}

    Figures~5 and 6 show the decay heat profile for PWR fuel discharged at 60~GWd/MT. At early cooling times, decay heat is 
    dominated by fission products such as Cs-137 and Sr-90. At longer cooling times, actinides—particularly Pu-238, Pu-241, 
    and Am-241—become the primary contributors. Higher burnup fuel retains more long-term decay heat, which has implications 
    for storage, transport, and disposal.

    \begin{table}[H]
        \begin{center}
            \includegraphics[width=1.0\textwidth]{figures/bwr_table.png}
        \end{center}
        \caption{BWR 50 GWd/MT Spent Nuclear Fuel Decay Heat}
    \end{table}

    \begin{figure}[H]
        \begin{center}
            \includegraphics[width=1.0\textwidth]{figures/bwr_plot.png}
        \end{center}
        \caption{BWR 50 GWd/MT Spent Nuclear Fuel Decay Heat}
    \end{figure}

    Figures~7 and 8 present corresponding decay heat characteristics for BWR fuel discharged at 50~GWd/MT. Similar to the PWR case, 
    short-term heat generation is controlled by fission products, while long-term heat output is driven by actinides. 
    These representative datasets are essential for evaluating engineered storage systems and for bounding performance 
    assessments in fuel cycle simulations.\\

    While commercial reactors generate the vast majority of SNF in the United States, the DOE manages a separate and highly diverse 
    inventory originating from research, defense, and demonstration reactors, which is discussed in the following section.

%%%%%%%%%%%%%%%%%%%%%%%%%%%%%%%%%%%%%%%%%%%%%%%%%%%%%%%%%%%%%%%%%%%%%%%%%%%%%%%%%%%%%%%%%%%%%%%%%%%%%%%%%%%%%%%%%%%%%%%%%%%%%%%%%%%%%%%%%%%%%%%%%%%%%%%%%%%%
%%%%%%%%%%%%%%%%%%%%%%%%%%%%%%%%%%%%%%%%%%%%%%%%%%%%%%%%%%%%%%%%%%%%%%%%%%%%%%%%%%%%%%%%%%%%%%%%%%%%%%%%%%%%%%%%%%%%%%%%%%%%%%%%%%%%%%%%%%%%%%%%%%%%%%%%%%%%
\section{SNF at DOE Locations}

    The U.S. Department of Energy (DOE) is responsible for spent nuclear fuel (SNF) that falls outside the 
    commercial nuclear power sector. This includes fuel from defense production reactors, test and research reactors operated 
    across national laboratories, early demonstration power reactors from the Atomic Energy Commission era, university research 
    reactors once transferred to federal custody, and fuel used in research, examination, or fuel-cycle development programs. 
    DOE also manages SNF originating from naval nuclear propulsion, although naval fuel is tracked separately due to classification 
    restrictions and highly specialized handling requirements.\\

    Unlike the relatively standardized commercial LWR inventory, DOE’s SNF inventory is highly heterogeneous. 
    It includes fuel from graphite-moderated reactors, sodium-cooled reactors, high-flux isotope production reactors, 
    heavy-water reactors, TRIGA research reactors, breeder reactors, and early prototype power reactors. Fuel enrichments 
    span a wide range—from depleted uranium for production reactors to highly enriched uranium exceeding 93\% U-235 for 
    test and training reactors. Many reactors used bespoke fuel element geometries and materials, resulting in over 229,000 
    individual fuel elements or pieces, covering several hundred unique fuel types. This diversity is central to DOE’s ongoing 
    challenges in characterization, consolidation, and long-term disposal planning.\\

    By the end of 2022, DOE managed approximately 2,446~MTHM of SNF (excluding naval fuel), as documented in the Spent Fuel Database
     (SFD). To support system-level evaluations, this inventory has been grouped into 34 standardized categories based on 
     fuel matrix, cladding type, cladding condition, and enrichment. These parameters influence corrosion behavior, 
     radionuclide release, storage system performance, and repository compatibility. Each category aggregates fuel with similar 
     physical and chemical characteristics, enabling more tractable modeling of a highly diverse inventory.\\

    Radionuclide inventories for DOE-managed fuel are estimated through template depletion models that represent typical fuel 
    behavior at different burnups and cooling times. These templates are then scaled to match the mass, enrichment, and operating 
    history of individual SNF records. Based on these models, the total expected radioactivity of DOE-managed SNF in 2030 is 
    approximately 96~million~Ci for the nominal case. A bounding case, intended to capture the highest-burnup or least 
    well-characterized assemblies, reaches approximately 195~million~Ci. These values emphasize the wide variability in DOE SNF 
    and the presence of small subsets of high-activity fuel that dominate radiological behavior.

    \begin{table}[H]
        \begin{center}
            \includegraphics[width=1.0\textwidth]{figures/doe_projection.png}
        \end{center}
        \caption{Projected decay heat distribution of DOE spent nuclear fuel canisters for the year 2030. Most DOE canisters generate less than 300~W and over 60\% generate less than 100~W, indicating that the majority of DOE SNF is relatively low-power at long cooling times.}
    \end{table}

    In 1995, DOE made the strategic decision to consolidate SNF storage at three primary sites: the Hanford Site in Washington, 
    Idaho National Laboratory (INL) in Idaho, and the Savannah River Site (SRS) in South Carolina. These locations now host nearly 
    all DOE-managed SNF. Hanford contains the largest share, approximately 2,128~MTHM, primarily stored in sealed stainless-steel 
    Multi-Canister Overpacks (MCOs). INL stores approximately 270~MTHM, much of it originating from research and test reactors. 
    SRS holds approximately 27~MTHM, largely from isotope production reactors and research reactor returns. Although DOE has 
    established standard disposal canister designs—intended to support eventual geologic disposal—no DOE SNF has yet been packaged 
    into these standardized canisters.\\

    DOE's inventory also includes 173.6~MTHM of fuel originating from commercial nuclear power reactors but transferred to DOE 
    for research, testing, or fuel-cycle demonstration programs. This category includes the core debris from Three Mile Island 
    Unit~2, high-temperature gas-cooled reactor fuel from the Fort St. Vrain reactor, and various LWR assemblies used in 
    post-irradiation examination, rod consolidation trials, and vitrification studies. Much of this fuel is no longer intact 
    and must be packaged in specialized canisters prior to disposal.

    \begin{table}[H]
        \begin{center}
            \includegraphics[width=1.0\textwidth]{figures/doe.png}
        \end{center}
        \caption{Decay heat characteristics of canisters containing SNF of commercial (NPR) origin managed by DOE. The distribution reflects both intact LWR assemblies and non-intact debris packaged for long-term storage.}
    \end{table}

    In addition to research and commercial-origin fuel, DOE manages naval reactor SNF through the Naval Nuclear Propulsion 
    Program (NNPP). This fuel is discharged from nuclear-powered submarines, aircraft carriers, land-based prototype reactors, 
    and moored training ships. Naval fuel is fabricated with highly enriched uranium, resulting in exceptionally low transuranic 
    production compared to commercial LWR fuel. The current inventory is approximately 40~MTHM, with projections indicating less 
    than 65~MTHM by 2035. Naval SNF is stored in specially engineered canisters—either “naval short” or “naval long” 
    variants—designed to meet repository heat, shielding, and criticality constraints. As of 2022, 201 naval SNF canisters have 
    been loaded and are stored at INL.

    \begin{table}[H]
        \begin{center}
            \includegraphics[width=1.0\textwidth]{figures/naval.png}
        \end{center}
        \caption{Decay heat characteristics of naval SNF canisters. Although naval fuel is highly enriched, long-term decay heat is generally modest due to low transuranic buildup.}
    \end{table}

    Overall, DOE’s SNF inventory is characterized by its diversity, historical complexity, and unique storage challenges. 
    While commercial LWR fuel dominates total mass in the United States, DOE fuel dominates in terms of variety, specialization, 
    and the need for tailored treatment and disposal strategies. This distinction is critical for national-level fuel cycle 
    assessments and informs the development of repository design criteria and waste acceptance requirements.

%%%%%%%%%%%%%%%%%%%%%%%%%%%%%%%%%%%%%%%%%%%%%%%%%%%%%%%%%%%%%%%%%%%%%%%%%%%%%%%%%%%%%%%%%%%%%%%%%%%%%%%%%%%%%%%%%%%%%%%%%%%%%%%%%%%%%%%%%%%%%%%%%%%%%%%%%%%%
%%%%%%%%%%%%%%%%%%%%%%%%%%%%%%%%%%%%%%%%%%%%%%%%%%%%%%%%%%%%%%%%%%%%%%%%%%%%%%%%%%%%%%%%%%%%%%%%%%%%%%%%%%%%%%%%%%%%%%%%%%%%%%%%%%%%%%%%%%%%%%%%%%%%%%%%%%%%
\section{SNF at Other Sites}

    Beyond commercial nuclear power reactors and DOE facilities, a small amount of spent nuclear fuel (SNF) is located at 
    other sites throughout the United States. These holdings are comparatively minor—totaling only about 1.35~metric tons of 
    heavy metal (MTHM)—yet they represent a diverse set of institutions and reactor types that contribute to the broader national 
    fuel cycle inventory. The SNF at these locations is cataloged in the DOE Spent Fuel Database and originates primarily from 
    university research reactors, government agency-operated reactors, and commercial research and development centers.\\

    University research reactors constitute the largest group within this category. Approximately twenty such reactors are 
    currently operating in the United States, typically at very low power levels ranging from less than one watt to approximately
    10~MW, depending on reactor design and research application. Because these reactors use fuel at much lower power densities and 
    achieve much smaller burnups than commercial reactors, refueling events are infrequent. In many cases, entire reactor lifetimes 
    may pass without the need for a fuel discharge. When spent fuel is eventually removed from university research reactors, 
    it is transferred to DOE custody—most commonly to the Idaho National Laboratory (INL) or the Savannah River Site (SRS)—where 
    it becomes part of the broader DOE-managed SNF inventory discussed earlier. A small number of specialized critical facilities, 
    such as the AGN-201 reactors or the Rensselaer Polytechnic Institute Critical Facility, operate at such low power that they are 
    not expected to generate any spent fuel for discharge.

    \begin{table}[H]
        \begin{center}
            \includegraphics[width=1.0\textwidth]{figures/universities.png}
        \end{center}
        \caption{University research reactor sites and their associated SNF inventories. Although these facilities represent a wide geographic distribution, the total quantity of SNF remains very small compared to commercial or DOE sources.}
    \end{table}

    In addition to university-operated facilities, several research reactors are run by other government agencies. 
    These include national defense, materials testing, and specialized experimental reactors that support federal research programs. 
    Similar to university reactors, permanently discharged fuel from these facilities is shipped to INL or SRS, where it is 
    incorporated into DOE’s consolidated inventory. Table~4.2 summarizes these government-operated research reactors and the 
    corresponding SNF quantities that have been transferred to DOE management.

    \begin{table}[H]
        \begin{center}
            \includegraphics[width=1.0\textwidth]{figures/government.png}
        \end{center}
        \caption{Research reactors operated by other government agencies and the associated SNF transferred to DOE custody for long-term management.}
    \end{table}

    A final category within this group consists of commercial research and development centers, a small number of which operate or 
    historically operated research reactors. These facilities contribute modest quantities of SNF, and in some cases, such as the 
    BWX Technologies site in Virginia, they support fuel-cycle research and destructive examination activities rather than 
    power-producing reactor operations. Table~4.3 lists these commercial R\&D centers and their corresponding SNF inventories. 
    Although the total mass of fuel generated by these institutions is small, their presence reflects the broad technical 
    landscape of nuclear research in the United States and the need for DOE to manage diverse and specialized fuel forms 
    originating from non-power, non-defense applications.

    \begin{table}[H]
        \begin{center}
            \includegraphics[width=1.0\textwidth]{figures/commercial_research.png}
        \end{center}
        \caption{Commercial research and development centers with spent nuclear fuel inventories. These sites include both research reactors and facilities that conduct fuel examination and testing.}
    \end{table}

    Overall, the SNF at these other sites represents the smallest contribution to the national fuel inventory by mass, 
    but it remains important due to its broad diversity and the unique characteristics of the research-reactor fuels involved. 
    When transferred to DOE custody, this fuel must be integrated into the same long-term planning, storage frameworks, 
    and disposition pathways used for the larger DOE SNF inventory.

%%%%%%%%%%%%%%%%%%%%%%%%%%%%%%%%%%%%%%%%%%%%%%%%%%%%%%%%%%%%%%%%%%%%%%%%%%%%%%%%%%%%%%%%%%%%%%%%%%%%%%%%%%%%%%%%%%%%%%%%%%%%%%%%%%%%%%%%%%%%%%%%%%%%%%%%%%%%
%%%%%%%%%%%%%%%%%%%%%%%%%%%%%%%%%%%%%%%%%%%%%%%%%%%%%%%%%%%%%%%%%%%%%%%%%%%%%%%%%%%%%%%%%%%%%%%%%%%%%%%%%%%%%%%%%%%%%%%%%%%%%%%%%%%%%%%%%%%%%%%%%%%%%%%%%%%%
\section{Reprocessing Waste}


    In addition to spent nuclear fuel (SNF), the United States manages a substantial inventory of high-level radioactive waste (HLW) 
    produced through reprocessing activities. Reprocessing refers to the chemical separation of uranium, plutonium, and fission 
    products from irradiated nuclear fuel. Although commercial reprocessing ended decades ago in the United States, 
    large quantities of waste were generated during both defense missions and early nuclear technology development. 
    These materials remain an important component of the national waste inventory and are distributed across several DOE sites, 
    as well as one former commercial facility at West Valley, New York.\\

    The physical form of reprocessing waste varies depending on the treatment processes used. 
    Historically, reprocessing operations produced large volumes of liquid HLW stored in underground tanks. 
    Long-term isolation of these wastes requires stabilization, and two principal methods have been used: vitrification and calcination. 
    Vitrification involves blending liquid waste with glass-forming additives and melting the mixture into a durable borosilicate glass 
    that is poured into stainless-steel canisters. Calcination, used at the Idaho National Laboratory (INL), converts liquid waste into 
    a granular, powder-like solid through high-temperature evaporation. In addition to these aqueous waste forms, INL also manages metal 
    and ceramic waste streams produced through electrochemical treatment of sodium-bonded SNF, a specialized fuel type used in early 
    fast-reactor programs.\\

    Among the DOE sites, the Savannah River Site (SRS) has made the most progress in vitrification. Since 1996, the Defense Waste 
    Processing Facility has immobilized HLW into standardized glass canisters, producing 4,346 canisters as of December~31,~2022. 
    INL, by contrast, generated roughly 30,000~m\textsuperscript{3} of liquid waste during past reprocessing operations; between 
    1960 and 1997, this material was converted into approximately 4,400~m\textsuperscript{3} of calcine, which is currently stored 
    in stainless-steel bins housed within concrete vaults. At the Hanford Site, reprocessing during defense plutonium production 
    generated an even larger inventory: approximately 56~million gallons of liquid HLW. About 220,000~m\textsuperscript{3} of this 
    material remains in underground tanks, and a major vitrification facility is under construction to process the waste into glass 
    for long-term disposition.\\

    Hanford also stores 1,936 cesium and strontium capsules produced between 1974 and 1985. Originally removed from liquid waste to 
    support isotope applications, these capsules contain high concentrations of fission products and historically accounted for 
    nearly 109~million curies of radioactivity. Through natural decay, the inventory has decreased to about 42~million curies, 
    and continued decay will further reduce the heat output and radiological hazard over the coming decades. DOE is currently 
    preparing new dry-storage systems to replace the aging water-filled storage basins that house these capsules.\\

    Electrochemical treatment at INL provides a distinct pathway for stabilizing sodium-bonded fuel from early fast-reactor programs. 
    This process separates uranium metal from fission products and residual materials, generating two waste streams: a ceramic waste 
    form containing salts and active fission products, and a metallic waste form containing cladding hulls and noble metals. 
    Up to 60~MTHM of sodium-bonded fuel may ultimately be treated, and about 4~MTHM has been processed to date.\\

    The overall radionuclide inventory associated with DOE reprocessing waste corresponds to an estimated 1.3~million watts of 
    decay heat. Most vitrified waste canisters—over 99\%—are projected to generate less than 1~kW of thermal power, reflecting the 
    durability and stability of the immobilized waste forms. Although the total number of canisters may change as waste treatment 
    continues, the radioactive content of the waste will decrease only through radioactive decay.\\

    The only commercial reprocessing facility ever operated in the United States was located at West Valley, 
    New York, where Nuclear Fuel Services processed approximately 640~MTHM of SNF between 1966 and 1972. 
    This operation generated about 2,500~m\textsuperscript{3} of liquid HLW, which was vitrified between 1996 and 2001 into 
    278 canisters. These canisters, including two used to drain the processing melter and one non-routine HLW canister, remain 
    stored on site. Although the total waste mass is small compared to DOE’s defense-related inventory, the West Valley canisters 
    remain an important legacy of the nation’s early commercial nuclear fuel cycle.

%%%%%%%%%%%%%%%%%%%%%%%%%%%%%%%%%%%%%%%%%%%%%%%%%%%%%%%%%%%%%%%%%%%%%%%%%%%%%%%%%%%%%%%%%%%%%%%%%%%%%%%%%%%%%%%%%%%%%%%%%%%%%%%%%%%%%%%%%%%%%%%%%%%%%%%%%%%%
%%%%%%%%%%%%%%%%%%%%%%%%%%%%%%%%%%%%%%%%%%%%%%%%%%%%%%%%%%%%%%%%%%%%%%%%%%%%%%%%%%%%%%%%%%%%%%%%%%%%%%%%%%%%%%%%%%%%%%%%%%%%%%%%%%%%%%%%%%%%%%%%%%%%%%%%%%%%

\section{Incorporating the National SNF Inventory into Cyclus Fuel Cycle Modeling}

    The detailed characterization of the U.S. spent nuclear fuel (SNF) inventory developed in this report provides a foundation for 
    quantitative fuel-cycle modeling using the Cyclus simulation framework. Cyclus is an open-source, agent-based nuclear 
    fuel cycle simulator that enables the construction of user-defined fuel cycle systems through modular facilities 
    (“archetypes”). Each facility implements specific behaviors—such as storage, processing, transformation, 
    or consumption of nuclear materials—allowing complex fuel cycle scenarios to be assembled from flexible, 
    interoperable components. The comprehensive summary of national SNF presented in this report directly informs the 
    development of new Cyclus modules and scenarios designed to evaluate accelerator-driven system (ADS) configurations 
    and examine their potential impact on the U.S. waste inventory.\\

    A central objective of this work is to quantify the fuel-cycle performance of XDS technologies at isotopic resolution. 
    Achieving this requires realistic input streams that reflect the current and projected isotopic compositions, burnup histories, 
    and decay-corrected radionuclide inventories of spent fuel discharged from commercial reactors. Using the updated 
    STANDARDS 5.0.1 dataset (formerly UNF-ST\&DARDS), aged-corrected isotopic vectors are used to construct representative SNF 
    compositions that capture the effects of burnup, initial enrichment, fuel type, and decay time. 
    These data form the basis for establishing XDS input streams and for disaggregating the national SNF inventory into resource 
    classes that vary in transuranic content, radiotoxicity, or suitability for transmutation. In this way, 
    Cyclus simulations can distinguish which portions of the national inventory hold the highest potential value for MA-burning 
    ADS concepts.\\

    To support these objectives, a dedicated U.S. inventory module is being developed within Cyclus. This module is designed to 
    represent the national waste inventory as a dynamic storage-type facility that holds SNF categorized and parameterized 
    according to the latest ST\&DARDS dataset. Within the module, each fuel batch is stored as a material object with an 
    associated mass, isotopic composition, and decay history. The module enables external Cyclus facilities—such as 
    ADS reactors—to request material based on user-defined specifications. When material is supplied, the inventory is 
    automatically updated to reflect removal of the requested mass and isotopics. This functionality allows scenario 
    studies to explore how much of the national inventory could be processed, the rate at which materials are consumed, 
    and how inventory composition evolves over time as different technologies are deployed.\\

    Parallel to the inventory module, an ADS facility module is under development to represent accelerator-driven systems 
    capable of consuming transuranic-rich SNF. This module is structured to accommodate a range of ADS design concepts, 
    including different target materials, coolant types, and fuel compositions. The first implementation is based on the 
    ADS configuration proposed at Argonne National Laboratory. Within Cyclus, the ADS module receives material from the 
    U.S. inventory module, processes it according to user-specified operating parameters, and tracks the resulting mass reduction, 
    changes in isotopic composition, and key performance metrics such as minor actinide destruction rate, energy production, 
    and waste radiotoxicity reduction. The combination of these two modules—one representing the national inventory and 
    the other representing consumption pathways—enables system-level evaluation of how ADS deployment scales, how many 
    units are needed to achieve specific waste reduction goals, and how quickly the inventory could be transmuted under 
    different scenarios.\\

    Together, these modules create a quantitative framework linking the detailed SNF characterization performed in this report 
    with forward-looking fuel cycle analyses. By embedding accurate isotopic compositions derived from ST\&DARDS data into 
    Cyclus simulations, it becomes possible to evaluate XDS system designs, assess trade-offs among accelerator or target 
    configurations, compare transmutation strategies, and measure their aggregate impact on the U.S. waste inventory. 
    This integration ensures that future scenario analyses are grounded in realistic, data-driven representations of 
    the commercial and DOE-managed SNF resources, enabling robust evaluation of advanced fuel cycle strategies.

%%%%%%%%%%%%%%%%%%%%%%%%%%%%%%%%%%%%%%%%%%%%%%%%%%%%%%%%%%%%%%%%%%%%%%%%%%%%%%%%%%%%%%%%%%%%%%%%%%%%%%%%%%%%%%%%%%%%%%%%%%%%%%%%%%%%%%%%%%%%%%%%%%%%%%%%%%%%
%%%%%%%%%%%%%%%%%%%%%%%%%%%%%%%%%%%%%%%%%%%%%%%%%%%%%%%%%%%%%%%%%%%%%%%%%%%%%%%%%%%%%%%%%%%%%%%%%%%%%%%%%%%%%%%%%%%%%%%%%%%%%%%%%%%%%%%%%%%%%%%%%%%%%%%%%%%%


\section{Conclusion}

    The national inventory of spent nuclear fuel (SNF) reflects more than six decades of commercial power generation, 
    research reactor operations, defense missions, and early fuel-cycle development in the United States. 
    The vast majority of this inventory originates from commercial light-water reactors, which have discharged 
    approximately 91,000~MTU of fuel through the end of 2022. These assemblies are stored almost entirely at reactor sites, 
    either in spent fuel pools or in independent spent fuel storage installations. Their characteristics—particularly the steady 
    historical increase in fuel enrichment and burnup—strongly shape isotopic composition, decay heat behavior, and long-term storage 
    and disposal requirements. As burnup approaches 50--60~GWd/MT for typical PWR fuel, the isotopic vectors relevant to depletion 
    modeling, repository design, and criticality evaluations continue to evolve in ways that require consistent, data-driven 
    characterization.\\

    In parallel to the commercial sector, the DOE manages a smaller but significantly more heterogeneous SNF inventory of 
    roughly 2,446~MTU. This category encompasses materials from defense production reactors, national laboratory test reactors, 
    early demonstration power reactors, university and government research programs, and high-enrichment research reactor fuels. 
    These diverse fuel forms span a wide range of enrichments, cladding materials, and fabrication methods, and therefore play a 
    distinct role in system-level evaluations of storage, transportation, and disposal. Additional small quantities of SNF—totaling 
    about 1.35~MTU—remain at university and commercial research facilities, but all such material ultimately transitions into DOE 
    custody for long-term management.\\

    Reprocessing waste constitutes another significant element of the national inventory. At the Savannah River Site, 4,346 vitrified 
    waste canisters have been produced through decades of HLW treatment, while the West Valley Demonstration Project has 
    generated an additional 278 vitrified canisters from past commercial reprocessing operations. Together with calcined waste at 
    the Idaho National Laboratory and liquid HLW stored at Hanford, these materials represent the legacy of earlier reprocessing 
    activities and highlight the complexity of stabilizing chemically diverse and highly radioactive waste streams for geological 
    disposal.\\

    Taken together, these inventories illustrate the technical breadth and historical depth of spent fuel management in the United 
    States. They also underscore the importance of accurate, assembly-level isotopic data for fuel cycle modeling, decay heat 
    estimation, criticality safety, and waste package design. As future analyses depend increasingly on realistic depletion behavior 
    and burnup-dependent material compositions, the systematic characterization summarized in this report provides a necessary 
    foundation for national fuel-cycle assessments and future repository planning.







